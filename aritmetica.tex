\documentclass[        
    a4paper,          % Tamanho da folha A4.
    sumario = tradicional, % Estilo tradicional de sumários do memoir.
    12pt,             % Tamanho da fonte 12pt.
    chapter=TITLE,    % Todos os capitulos devem ter caixa alta.
    section=TITLE,    % Todas as secoes devem ter caixa alta.
    oneside,          % Usada para impressao em apenas uma face do papel.
    english,          % Hifenizacoes em ingles.
    spanish,          % Hifenizacoes em espanhol.
    brazil            % Ultimo idioma eh o idioma padrao do documento.
]{abntex2}\usepackage[]{graphicx}\usepackage[]{color}
% maxwidth is the original width if it is less than linewidth
% otherwise use linewidth (to make sure the graphics do not exceed the margin)
\makeatletter
\def\maxwidth{ %
  \ifdim\Gin@nat@width>\linewidth
    \linewidth
  \else
    \Gin@nat@width
  \fi
}
\makeatother

\definecolor{fgcolor}{rgb}{0.345, 0.345, 0.345}
\newcommand{\hlnum}[1]{\textcolor[rgb]{0.686,0.059,0.569}{#1}}%
\newcommand{\hlstr}[1]{\textcolor[rgb]{0.192,0.494,0.8}{#1}}%
\newcommand{\hlcom}[1]{\textcolor[rgb]{0.678,0.584,0.686}{\textit{#1}}}%
\newcommand{\hlopt}[1]{\textcolor[rgb]{0,0,0}{#1}}%
\newcommand{\hlstd}[1]{\textcolor[rgb]{0.345,0.345,0.345}{#1}}%
\newcommand{\hlkwa}[1]{\textcolor[rgb]{0.161,0.373,0.58}{\textbf{#1}}}%
\newcommand{\hlkwb}[1]{\textcolor[rgb]{0.69,0.353,0.396}{#1}}%
\newcommand{\hlkwc}[1]{\textcolor[rgb]{0.333,0.667,0.333}{#1}}%
\newcommand{\hlkwd}[1]{\textcolor[rgb]{0.737,0.353,0.396}{\textbf{#1}}}%
\let\hlipl\hlkwb

\usepackage{framed}
\makeatletter
\newenvironment{kframe}{%
 \def\at@end@of@kframe{}%
 \ifinner\ifhmode%
  \def\at@end@of@kframe{\end{minipage}}%
  \begin{minipage}{\columnwidth}%
 \fi\fi%
 \def\FrameCommand##1{\hskip\@totalleftmargin \hskip-\fboxsep
 \colorbox{shadecolor}{##1}\hskip-\fboxsep
     % There is no \\@totalrightmargin, so:
     \hskip-\linewidth \hskip-\@totalleftmargin \hskip\columnwidth}%
 \MakeFramed {\advance\hsize-\width
   \@totalleftmargin\z@ \linewidth\hsize
   \@setminipage}}%
 {\par\unskip\endMakeFramed%
 \at@end@of@kframe}
\makeatother

\definecolor{shadecolor}{rgb}{.97, .97, .97}
\definecolor{messagecolor}{rgb}{0, 0, 0}
\definecolor{warningcolor}{rgb}{1, 0, 1}
\definecolor{errorcolor}{rgb}{1, 0, 0}
\newenvironment{knitrout}{}{} % an empty environment to be redefined in TeX

\usepackage{alltt}

% Importações de pacotes
\usepackage[utf8]{inputenc}                         % Acentuação direta
\usepackage[T1]{fontenc}                            % Codificação da fonte em 8 bits
\usepackage{graphicx}                               % Inserir figuras
\usepackage{amsfonts, amssymb, amsmath}             % Fonte e símbolos matemáticos
\usepackage{booktabs}                               % Comandos para tabelas
\usepackage{verbatim}                               % Texto é interpretado como escrito no documento
\usepackage{multirow, array}                        % Múltiplas linhas e colunas em tabelas
\usepackage{indentfirst}                            % Endenta o primeiro parágrafo de cada seção.
\usepackage{listings}                               % Utilizar codigo fonte no documento
\usepackage{xcolor}
\usepackage{microtype}                              % Para melhorias de justificação?
\usepackage[portuguese,ruled,lined]{algorithm2e}    % Escrever algoritmos
\usepackage{algorithmic}                            % Criar Algoritmos  
%\usepackage{float}                                  % Utilizado para criação de floats
\usepackage{amsgen}
\usepackage{lipsum}                                 % Usar a simulação de texto Lorem Ipsum
\usepackage{titlesec}                               % Permite alterar os títulos do documento
\usepackage{tocloft}                                % Permite alterar a formatação do Sumário
\usepackage{etoolbox}                               % Usado para alterar a fonte da Section no Sumário
%\usepackage[nogroupskip,nonumberlist,acronym]{glossaries}                % Permite fazer o glossario
\usepackage{caption}                                % Altera o comportamento da tag caption
\usepackage[alf, abnt-emphasize=bf, bibjustif, recuo=0cm, abnt-etal-cite=3, abnt-etal-list=0,abnt-etal-text=it]{abntex2cite}  % Citações padrão ABNT
%\usepackage[bottom]{footmisc}                      % Mantém as notas de rodapé sempre na mesma posição
%\usepackage{times}                                 % Usa a fonte Times
\usepackage{mathptmx}                               % Usa a fonte Times New Roman										
%\usepackage{lmodern}                               % Usa a fonte Latin Modern
%\usepackage{subfig}                                % Posicionamento de figuras
%\usepackage{scalefnt}                              % Permite redimensionar tamanho da fonte
%\usepackage{color, colortbl}                       % Comandos de cores
%\usepackage{lscape}                                % Permite páginas em modo "paisagem"
%\usepackage{ae, aecompl}                           % Fontes de alta qualidade
%\usepackage{picinpar}                              % Dispor imagens em parágrafos
%\usepackage{latexsym}                              % Símbolos matemáticos
%\usepackage{upgreek}                               % Fonte letras gregas
\usepackage{appendix}                               % Gerar o apendice no final do documento
\usepackage{paracol}                                % Criar paragrafos sem identacao
\usepackage{lib/unirtex2-teste}		                    % Biblioteca com as normas da UECE para trabalhos academicos
\usepackage{pdfpages}                               % Incluir pdf no documento
\usepackage{amsmath}                                % Usar equacoes matematicas
\usepackage{hyperref}
\usepackage{xspace}
% Organiza e gera a lista de abreviaturas, simbolos e glossario
%\makeglossaries

% Gera o Indice do documento
%\usepackage{amsthm} % Pacote de ambiente proof.


%%%%%%%%%%%%%%%%%%%%%%%%%%%%%%%%%%%%%%%%%%%%%%%%%%%%%
%%          Configuracoes do UnirTeX2              %%
%%%%%%%%%%%%%%%%%%%%%%%%%%%%%%%%%%%%%%%%%%%%%%%%%%%%%

% Opcoes disponiveis

%\relatorioacademico{relatorio}
\trabalhoacademico{tccgraduacao}
%\trabalhoacademico{tccespecializacao}
%\trabalhoacademico{dissertacao} será retirado
%\trabalhoacademico{tese} será retirado

% Define se o trabalho eh uma qualificacao
% Coloque 'nao' para versao final do trabalho

%\ehqualificacao{nao}

% Remove as bordas vermelhas e verdes do PDF gerado
% Coloque 'sim' pare remover

\removerbordasdohyperlink{sim} 

% Adiciona a cor Azul a todos os hyperlinks

\cordohyperlink{nao}

%%%%%%%%%%%%%%%%%%%%%%%%%%%%%%%%%%%%%%%%%%%%%%%%%%%%%
%%          Informação sobre a IES                 %%
%%%%%%%%%%%%%%%%%%%%%%%%%%%%%%%%%%%%%%%%%%%%%%%%%%%%%

\ies{Fundação Universidade Federal de Rondônia}
\iessigla{UNIR}
\centro{Departamento de Matemática e Estatística}

%%%%%%%%%%%%%%%%%%%%%%%%%%%%%%%%%%%%%%%%%%%%%%%%%%%%%
%%        Informação para TCC de Graduacao         %%
%%%%%%%%%%%%%%%%%%%%%%%%%%%%%%%%%%%%%%%%%%%%%%%%%%%%%

\graduacaoem{Estatística}
\habilitacao{bacharelado} % Pode colocar tambem 'licenciada', 'bacharel'.

%%%%%%%%%%%%%%%%%%%%%%%%%%%%%%%%%%%%%%%%%%%%%%%%%%%%%
%%     Informação para TCC de Especializacao       %%
%%%%%%%%%%%%%%%%%%%%%%%%%%%%%%%%%%%%%%%%%%%%%%%%%%%%%

%\especializacaoem{Alfabetização de Crianças}

%%%%%%%%%%%%%%%%%%%%%%%%%%%%%%%%%%%%%%%%%%%%%%%%%%%%%
%%         Informação para Dissertacao             %%
%%%%%%%%%%%%%%%%%%%%%%%%%%%%%%%%%%%%%%%%%%%%%%%%%%%%%

%\programamestrado{Programa de Pós-Graduação em Ciência da Computação}
%\nomedomestrado{Mestrado Acadêmico em Ciência da Computação}
%\mestreem{Ciência da Computação}
%\areadeconcentracaomestrado{Ciência da Computação}

%%%%%%%%%%%%%%%%%%%%%%%%%%%%%%%%%%%%%%%%%%%%%%%%%%%%%
%%               Informação para Tese              %%
%%%%%%%%%%%%%%%%%%%%%%%%%%%%%%%%%%%%%%%%%%%%%%%%%%%%%

%\programadoutorado{Programa de Pós-Graduação em Saúde Coletiva}
%\nomedodoutorado{Doutorado em Saúde Coletiva}
%\doutorem{Saúde Coletiva}
%\areadeconcentracaodoutorado{Saúde Coletiva}

%%%%%%%%%%%%%%%%%%%%%%%%%%%%%%%%%%%%%%%%%%%%%%
%%  Informação relacionadas ao trabalho     %%
%%%%%%%%%%%%%%%%%%%%%%%%%%%%%%%%%%%%%%%%%%%%%%

\autor{Douglas Vinícius Gonçalves Araújo}
\titulo{Probabilidade e Aritmética}
\data{2020}
\local{Ji-Paraná - RO}

% Exemplo: \dataaprovacao{01 de Janeiro de 2012}
%\dataaprovacao{}

%%%%%%%%%%%%%%%%%%%%%%%%%%%%%%%%%%%%%%%%%%%%%
%%     Informação sobre o Orientador       %%
%%%%%%%%%%%%%%%%%%%%%%%%%%%%%%%%%%%%%%%%%%%%%

\orientador{Prof. Dr. Lucia}
\orientadories{Fundação Universidade Federal de Rondônia}
\orientadorcentro{Departamento de Matemática e Estatística}
\orientadorfeminino{sim} % Coloque 'sim' se for do sexo feminino

%%%%%%%%%%%%%%%%%%%%%%%%%%%%%%%%%%%%%%%%%%%%%
%%      Informação sobre o Co-orientador   %%
%%%%%%%%%%%%%%%%%%%%%%%%%%%%%%%%%%%%%%%%%%%%%

% Deixe o nome do coorientador em branco para remover do documento

%\coorientador{Prof.ª Dra. Roziane Sobreira dos Santos}
%\coorientadories{Fundação Universidade Federal de Rondônia}
%\coorientadorcentro{Departamento de Matemática e Estatística}
%\coorientadorfeminino{sim} % Coloque 'sim' se for do sexo feminino

%%%%%%%%%%%%%%%%%%%%%%%%%%%%%%%%%%%%%%%%%%%%%
%%      Informação sobre a banca           %%
%%%%%%%%%%%%%%%%%%%%%%%%%%%%%%%%%%%%%%%%%%%%%

% Atenção! Deixe o nome do membro da banca para remover da folha de aprovacao

% Exemplo de uso:
% \membrodabancadois{Prof. Dr. Fulano de Tal}
% \membrodabancadoisies{Universidade Estadual do Ceará - UECE}

%\membrodabancadois{Membro da Banca Dois}
%\membrodabancadoiscentro{Faculdade de Filosofia Dom Aureliano Matos – FAFIDAM}
%\membrodabancadoisies{Universidade do Membro da Banca Dois - SIGLA}
%\membrodabancatres{Membro da Banca Três}
%\membrodabancatrescentro{Centro de Ciências e Tecnologia - CCT}
%\membrodabancatresies{Universidade do Membro da Banca Três - SIGLA}
%\membrodabancaquatro{Membro da Banca Quatro}
%\membrodabancaquatrocentro{Centro de Ciências e Tecnologia - CCT}
%\membrodabancaquatroies{Universidade do Membro da Banca Quatro - SIGLA}
%\membrodabancacinco{Membro da Banca Cinco}
%\membrodabancacincocentro{Teste}
%\membrodabancacincoies{Universidade do Membro da Banca Cinco - SIGLA}
%\membrodabancaseis{Membro da Banca Seis}
%\membrodabancaseiscentro{}
%\membrodabancaseisies{Universidade do Membro da Banca Seis - SIGLA}
\IfFileExists{upquote.sty}{\usepackage{upquote}}{}
\begin{document}	

	% Elementos pré-textuais
	\imprimircapa
	\imprimirfolhaderosto{}
%	\imprimirfichacatalografica{elementos-pre-textuais/ficha-catalografica}
%	\imprimirerrata{elementos-pre-textuais/errata}
%	\imprimirfolhadeaprovacao
	\imprimirdedicatoria{elementos-pre-textuais/dedicatoria}
	\imprimiragradecimentos{elementos-pre-textuais/agradecimentos}
	\imprimirepigrafe{elementos-pre-textuais/epigrafe}
	\imprimirresumo{elementos-pre-textuais/resumo}
	\imprimirabstract{elementos-pre-textuais/abstract}
	\imprimirlistadeilustracoes
	\imprimirlistadetabelas
%	\imprimirlistadequadros
%	\imprimirlistadealgoritmos
%	\imprimirlistadecodigosfonte
%	\imprimirlistadeabreviaturasesiglas	
	\imprimirlistadesimbolos{elementos-pre-textuais/lista-de-simbolos}   
	\imprimirsumario
	
	%Elementos textuais
	\textual
	\chapter{INTRODUÇÃO}
\label{cap:introducao}








\section{OBJETIVOS}
\label{sec:objetivos}

\subsection{OBJETIVO GERAL}
\label{sec:objetivo-geral}




\subsection{OBJETIVO ESPECÍFICO}
\label{sec:objetivos-especificos}



\section{JUSTIFICATIVA}
\label{sec:justificativa}

	\chapter{REFERENCIAL TEÓRICO}
\label{cap:fundamentacao-teorica}

  \section{TEORIA DOS NÚMEROS}
  
    \subsection{Lógica Matemática}
    
    \subsection{Funções e Conjuntos}
    
    \subsection{Teoria dos Números}
    
      \subsubsection{Divisibilidade}
      
      \subsubsection{Princípio da Indução Finita}

%	\chapter{Trabalhos Relacionados}
\label{cap:trabalhos-relacionados}

Integer non lacinia magna. Aenean tempor lorem tellus, non sodales nisl commodo ut. Proin mattis placerat risus sit amet laoreet. Praesent sapien arcu, maximus ac fringilla efficitur, vulputate faucibus sem. Donec aliquet velit eros, sit amet elementum dolor pharetra eget. Integer eget mattis libero

\section{Trabalho Relacionado A}
\label{sec:trabalho-relacionado-a}

\lipsum[10]

	\begin{figure}[h!]
		\centering
		\Caption{\label{fig:exemplo-1} Lorem ipsum dolor sit amet, consectetur adipiscing elit. Suspendisse commodo lectus et augue elementum varius.}	
		\UECEfig{}{
			\fbox{\includegraphics[width=8cm]{figuras/figura-1}}
		}{
			\Fonte{Elaborado pelo autor}
		}	
	\end{figure}
	
\lipsum[11]

\section{Trabalho Relacionado B}
\label{sec:trabalho-relacionado-b}

Integer non lacinia magna. Aenean tempor lorem tellus, non sodales nisl commodo ut. Proin mattis placerat risus sit amet laoreet. Praesent sapien arcu, maximus ac fringilla efficitur, vulputate faucibus sem. Donec aliquet velit eros, sit amet elementum dolor pharetra eget. Integer eget mattis libero. Praesent ex velit, pulvinar at massa vel, fermentum dictum mauris. Ut feugiat accumsan augue, et ultrices ipsum euismod vitae

	\begin{figure}[h!]
		\centering
		\Caption{\label{fig:exemplo-2} Maecenas luctus augue odio, sed tincidunt nunc posuere nec}	
		\UECEfig{}{
			\fbox{\includegraphics[width=8cm]{figuras/figura-2}}
		}{
			\Fonte{Elaborado pelo autor}			
		}	
	\end{figure}

Nunc ac pretium dui. Mauris aliquam dapibus nulla ac mattis. Aenean non tortor volutpat, varius lectus vitae, accumsan nibh. Cras pretium vestibulum enim, id ullamcorper tortor ultrices non. Integer sodales viverra faucibus. Curabitur at dui lacinia, rhoncus lacus at, blandit metus. Integer scelerisque non enim quis ornare.

	\begin{quadro}[h!]	
		\centering
		\Caption{\label{qua:exemplo-1} Praesent ex velit, pulvinar at massa vel, fermentum dictum mauris. Ut feugiat accumsan augue}		
		\UECEqua{}{
			\begin{tabular}{|c|c|l|l|}
				\hline
				Quisque & pharetra & tempus & vulputate \\
				\hline
				E1 & Complete coverage by a single transcript & Both  & Complete\\
				\hline
				E2 & Complete coverage by more than & Both splice sites & Complete\\
				\hline
				E3 & Partial coverage & Both splice sites & Both \\				
				\hline
			\end{tabular}
		}{
			\Fonte{Elaborado pelo autor}
		}
	\end{quadro}
	
\lipsum[20]

	
	\begin{quadro}[h!]	
		\centering
		\Caption{\label{qua:exemplo-2} Duis faucibus, enim quis tincidunt pellentesque}		
		\UECEqua{}{
			\begin{tabular}{|c|c|}
				\hline
				Quisque & pharetra \\
				\hline
				E1 & Complete coverage by a single transcript \\
				\hline
				E2 & Complete coverage by more than \\
				\hline
				E3 & Partial coverage \\
				\hline
				E4 & Partial coverage \\
				\hline
				E5 & Partial coverage \\
				\hline
				E6 & Partial coverage \\
				\hline
				E7 & Partial coverage \\
				\hline
			\end{tabular}
		}{
			\Fonte{Elaborado pelo autor}
		}
	\end{quadro}

\lipsum[21]

Integer non lacinia magna. Aenean tempor lorem tellus, non sodales nisl commodo ut. Proin mattis placerat risus sit amet laoreet. Praesent sapien arcu, maximus ac fringilla efficitur, vulputate faucibus sem. Donec aliquet velit eros, sit amet elementum dolor pharetra eget. Integer eget mattis libero.
\Gls{ambiguidade}
\Gls{braile}
\Gls{coerencia}
\Gls{dialetos}
\Gls{elipse}
\Gls{locucao-adjetiva}
\Gls{modificadores}
\Gls{paronimos}
\Gls{sintese}
\Gls{borboleta}

	\chapter{METODOLOGIA}
\label{chap:metodologia}

Testando a citação... \cite{lamport1986latex}

 \chapter{CRONOGRAMA}
	
O cronograma do trabalho encontra-se na Tabela \ref{tab:table1}, onde as etapas são:

\begin{enumerate}
	\item Revisão bibliográfica;
	\item Obtenção do Material e Leitura;
	\item Obtenção dos dados;
	\item Tratamento dos dados;
	\item Escrever Introdução;
	\item Escrever Referencial Teórico;
	\item Escrever a Metodologia;
	\item Aplicação da metodologia;
	\item Análise final e conclusão;
	\item Revisão;
	\item Envio para conferência do orientador;
	\item Divulgação dos resultados - DEFESA PÚBLICA.
\end{enumerate}

\begin{table}[!htb]
	\centering
	\begin{tabular}{|c|c|c|c|c|c|c|c|c|c|c|c|c|}
		\hline
		& \multicolumn{12}{|c|}{Meses} \\
		\cline{2-13} Atividades	& \multicolumn{12}{|c|}{2020} \\
		\cline{2-13} & JAN & FEV & MAR & ABR & MAI & JUN & JUL & AGO & SET & OUT & NOV & DEZ \\
		\hline 1 & X & X & X & X & X & X & X & X & X & X & X & X \\
		\hline 2 & & &  &  & & & & & & & & \\
		\hline 3 & & &  &  & & & & & & & & \\
		\hline 4 & & &  &  & & & & & & & & \\
		\hline 5 & & &  &  & & & & & & & & \\
		\hline 6 & & &  &  & & & & & & & & \\
		\hline 7 & & &  &  & & & & & & & & \\
		\hline 8 & & &  &  & & & & & & & & \\
		\hline 9 & & &  &  & & & & & & & & \\
		\hline 10 & & &  &  & & & & & & & & \\
		\hline 11 & & &  &  & & & & & & & & \\
		\hline 12 & & &  &  & & & & & & & & \\
		\hline
	\end{tabular} \caption{\emph{Cronograma de Atividades}}
	\label{tab:table1}
\end{table}

%	\chapter{RESULTADOS E DISCUSSÕES}
\label{chap:resultados}

  \section{Lógica Matemática}

Conforme \cite{de2002iniciaccao} a lógica matemática adota como regra fundamentais do pensamento os dois seguintes princípios (ou axiomas):

\begin{axioma}
\textbf{PRINCÍPIO DA NÃO CONTRADIÇÃO:} \textit{uma proposição não pode ser verdadeira e falsa ao mesmo tempo.}
\end{axioma}

\begin{axioma}
\textbf{PRINCÍPIO DO TERCEIRO EXCLUÍDO:} \textit{toda proposição ou é verdadeira ou falsa, isto é, verifica-se sempre um destes casos e nunca um terceiro.}
\end{axioma}
  
Em virtude desses dois princípios temos que a Lógica Matemática é uma lógica bivalente. Por definição, proposição significa uma oração declarativa, que tem sentido afirmativo completo.  

    \subsection{Proposições Simples e Compostos}
    
Definição: chama-se de proposição simples ou proposições atômicas aquela que não contém nenhuma outra proposição como parte integrante de si mesma. Normalmente representado por letras latina minúsculas: \textit{p, q, r}.

\noindent\textbf{Exemplo:} O número 25 é quadrado perfeito.

Definição: chama-se proposição composta ou proposição molecular aquela formada pela combinação de duas ou mais proposições. Representado por letras latinas maiúsculas: \textit{P, Q, R}.

\noindent\textbf{Exemplo:} O número $\pi$ é irracional e maior do que 4.

Usualmente as proposições compostas que são formadas pela combinação das proposições simples \textit{p, q, r,...}, escreve-se: $P(p,q,r,...)$
    
    \subsection{Conectivos}

Definição: chama-se conectivos palavras ou expressões que se usam para formar, interligar novas proposições a partir de outras. São conectivos usuais em lógica as palavras: "e", "ou", "não", "se... então... ", "... se e somente se...".

		\subsubsection{Negação}
		
\textbf{Definição:} negação de uma preposição \textit{p} representada por "não \textit{p}" cujo o valor lógico é verdadeiro quando \textit{p} é falso e vice-verso.

\begin{center}		
\begin{tabular}{@{ }c | c@{ }@{ }c}
	p & $\sim$ & p\\
	\hline 
	V & F & \\
	F & V & \\
\end{tabular}
\end{center}

		\subsubsection{Conjunção}

\textbf{Definição:} a conjunção de duas proposições \textit{p} e \textit{q} a proposição representada por "\textit{p} e \textit{q}, cujo o valor lógico é a verdade (\textbf{V}) quando ambas proposições são verdadeiras e falsa nos demais casos.

\begin{center}
\begin{tabular}{@{ }c@{ }@{ }c | c@{}@{ }c@{ }@{ }c@{ }@{ }c@{ }@{}c@{ }}
	p & q &  & p & $\land$ & q & \\
	\hline 
	V & V &  &  & V &  & \\
	V & F &  &  & F &  & \\
	F & V &  &  & F &  & \\
	F & F &  &  & F &  & \\
\end{tabular} 
\end{center}
	
		\subsubsection{Disjunção}

\textbf{Definição:} a disjunção de duas proposições \textit{p} e \textit{q}, representadas por "\textit{p} ou \textit{q}", cujo o valor lógico é a verdade quando ao menos uma das proposições é verdadeira e falsa quando ambas as preposições são falsas.

\begin{center}
\begin{tabular}{@{ }c@{ }@{ }c | c@{}@{ }c@{ }@{ }c@{ }@{ }c@{ }@{}c@{ }}
	p & q &  & p & $\lor$ & q & \\
	\hline 
	V & V &  &  & V &  & \\
	V & F &  &  & V &  & \\
	F & V &  &  & V &  & \\
	F & F &  &  & F &  & \\
\end{tabular}
\end{center}

		\subsubsection{Disjunção Exclusiva}
		
\textbf{Definição:} disjunção exclusiva de duas proposições \textit{p} e \textit{q} representada por "ou \textit{p} ou \textit{q}" ou "ou \textit{p} ou \textit{q}, mas não ambas", cujo o valor lógico é a verdade somente quando \textit{p} é verdadeiro ou \textit{q} é verdadeiro

\begin{center}
\begin{tabular}{@{ }c@{ }@{ }c | c@{}@{ }c@{ }@{ }c@{ }@{ }c@{ }@{}c@{ }}
	p & q &  & p & $\veebar$ & q & \\
	\hline 
	V & V &  &  & F &  & \\
	V & F &  &  & V &  & \\
	F & V &  &  & V &  & \\
	F & F &  &  & F &  & \\
\end{tabular}
\end{center}

		\subsubsection{Condicional}

\noindent \textbf{Definição:} Proposição condicional representado por "se p, então q, cujo o valor lógico é falsidade no caso que p é verdadeira e q é falso e verdade nos demais casos. Os valores lógicos de duas proposições, definido pela seguinte tabela-verdade:

\begin{center}
	\begin{tabular}{@{ }c@{ }@{ }c | c@{}@{ }c@{ }@{ }c@{ }@{ }c@{ }@{}c@{ }}
		p & q &  & p & $\to$ & q & \\
		\hline 
		V & V &  &  & V &  & \\
		V & F &  &  & F &  & \\
		F & V &  &  & V &  & \\
		F & F &  &  & V &  & \\
	\end{tabular}
\end{center}

Também se lê de uma das seguintes maneiras:

\begin{enumerate}[label=(\roman*)]
	\centering
	\item p é condição suficiente para q;
	\item q é condição necessária para p.
\end{enumerate}

Neste conectivo lógico é diz que p é o \textbf{antecedente} e q o \textbf{consequente}. O símbolo "$\to$" é chamado de símbolo de implicação.

		
		
		\subsubsection{Bicondicional}
		
\noindent \textbf{Definição:} Proposição bicondicional representada por "p se e somente se q", cujo o valor lógico é a verdade quando p e q são ambas verdadeiras ou ambas falsas, e falsidade nos demais casos. O valor lógico da bicondicional de duas proposições definidas pela seguinte tabela-verdade:

		

\begin{center}
	\begin{tabular}{@{ }c@{ }@{ }c | c@{}@{ }c@{ }@{ }c@{ }@{ }c@{ }@{}c@{ }}
		p & q &  & p & $\leftrightarrow$ & q & \\
		\hline 
		V & V &  &  & V &  & \\
		V & F &  &  & F &  & \\
		F & V &  &  & F &  & \\
		F & F &  &  & V &  & \\
	\end{tabular}
\end{center}

Simbolicamente, "$\Leftrightarrow$" também se lê de uma das seguintes maneiras:
\begin{enumerate}
	\item p é condição necessária e suficiente para q;
	\item q é condição necessária e suficiente para p.
\end{enumerate}
  
	\section{Conjuntos}
	


  	\section{Tipos de Demonstrações}
  
Resumidamente existem dois tipos de demonstrações em matemática: as demonstrações direta e demonstrações indiretas. Grande parte dos teoremas tem a estrutura "se \textit{P}, então \textit{Q}, onde \textit{P} e \textit{Q} são afirmações falsas ou verdadeiras. Chamamos de \textit{P} de hipótese ou premissa e \textit{Q} de tese ou conclusão.

    \subsection{Demonstrações Diretas}

O procedimento para uma demonstração direta é através de implicações lógicas encadeadas que levam $P$ diretamente a $Q$. Ou seja, de uma sentença $p \to q$ funciona da seguinte forma: assuma que o antecedente $p$ é verdade (hipótese) e deduza o consequente (tese) $q$.

\noindent \textbf{Exemplo 1:} Quaisquer dois quadrados perfeitos consecutivos diferem por um número ímpar.

\noindent \textbf{Demonstração:} Suponhamos que $a$ e $b$ sejam inteiros quadrados perfeitos consecutivos, ou seja, $a = n^2$ e $b = (n+1)^2$. Queremos mostrar que eles diferem por um número ímpar, ou seja, $b-a$ ou $a-b$ é um número ímpar. Como $a = n^2$ e $b = (n+1)^2$ então $b-a = (n+1)^2 - n^2 = n^2+2n+1-n^2 = 2n+1$. Portanto, $b-a$ é um número ímpar.

\noindent \textbf{Exemplo 2:} Sejam $A$ e $B$ conjuntos. Mostre que $A \subset B \Leftrightarrow \complement_B \subset \complement_A$.

\noindent \textbf{Demonstração:} Suponhamos que $A \subset B$. Então um elemento $x \in \complement_B$ não pode pertencer ao conjunto $B$, principalmente não pertencer a $A$. Logo $x \in \complement_B \Rightarrow x \in \complement_A$, ou seja, $\complement_B \subset \complement_A$. Semelhantemente pela propriedade reflexiva, $\complement(\complement_A) = A$, temos $\complement_B \subset \complement_A$ então, $\complement(\complement_A) \subset \complement(\complement_B)$, obtemos $A \subset B$.

    \subsection{Demonstrações Indiretas}

Existem dois tipos de demonstrações indireta que podemos usar para estabelecer uma afirmação condicional da forma $P \Rightarrow Q$: demonstração por absurdo e demonstração por contra-positiva.

\noindent \textbf{Demonstração por absurdo:} Na demonstração por absurdo nós assumimos que a hipótese $P$ é verdadeira, mas nesse caso supomos que a conclusão $Q$ é falsa. O objetivo é mostrar que a combinação da validade da hipótese $P$ com a não validade da tese $Q$ produz um resultado absurdo. Dessa forma segue que $Q$ é verdadeira.

\noindent \textbf{Exemplo:} Mostre que $\sqrt{2} \notin \mathbb{Q}$. Isso é o mesmo que dizer: Se $x \in \mathbb{R}, \ x > 0 $ e $ x^2 = 2$, então $x \notin \mathbb{Q}$.

\noindent \textbf{Demonstração:} Sabemos que um número $r \in \mathbb{R}$ é dito racional se existem inteiros $p, \ q$ sendo $q \neq 0$ tais que $r = \frac{p}{q}$.
  
  
    \section{Indução Finita e Boa Ordenação}

Uma propriedade básica dos números naturais e uma ferramenta indispensável na demonstração de muitos teoremas é: o Princípio da Indução Finita (PIF) que é divida em duas formas.

Seja $P(n)$ uma propriedade do número natural $n$, por exemplo:
\begin{itemize}
  \item $1+2+...+n=\frac{n(n+1)}{2}$;
  \item $1^2 + 2^2 + ... + n^2 = \frac{n(n+1)(2n+1)}{6}$;
  \item $1^3 + 2^3 + ... + n^3 = (1 + 2 + ... + n)^2$;
  \item $F_1 + F_2 + ... + F_n = F_{n+2} - 1$;
\end{itemize}

Para provar que a sentença aberta $P(n)$ é válida para todo natural $n \geq n_0$ é necessário utilizar o \textit{Princípio da Indução Finita} (PIF), que é um dos axiomas só o conjunto dos números naturais possuem. O PIF consiste em verificar duas etapas:
\begin{enumerate}
	\item (Base de Indução): $P(n_0)$ é verdadeira;
	\item (Passo Indutivo): Se $P(n)$ é verdadeira para algum número natural $n \geq n_0$, então $P(n+1)$ também é verdadeira.
\end{enumerate}

\begin{exem}\label{exemplo-ind1}
	Demonstrar que para todo inteiro positivo n,
\begin{equation*}
	1 + 2 + ... + n = \frac{n(n+1)}{2}
\end{equation*}
\end{exem}

\noindent Solução: Sabemos que para $P(1): 1 = \frac{1(1+1)}{2}$, onde a igualdade é verdadeira para $n = 1$ (base de indução). 
Suponhamos que seja verdadeiro para um $n = k$ (hipótese de indução):
\begin{equation*}
	1 + 2 + ... + k = \frac{k(k+1)}{2}.
\end{equation*}
\noindent Somando $k+1$ em ambos os lados da igualdade, obtemos
\begin{eqnarray*}
	1 + 2 + ... + k + (k+1) &=& \frac{k(k+1)}{2} + (k+1) \\
	&=& \frac{(k+1)(k+2)}{2},
\end{eqnarray*}
\noindent neste caso a igualdade também é válida para $n = k+1$. Pelo PIF, a igualdade vale para todo número natural $n \geq 1$.

\begin{exem}
	Demonstrar que para todo inteiro positivo n,
	\begin{equation*}
	1^2 + 2^2 + ... + k^2 = \frac{n(n+1)(2n+1)}{6}
	\end{equation*}
\end{exem}

\noindent Solução: Observamos que para $P(1): 1^2 = \frac{1(1+1)(2 \cdot 1 + 1)}{6}$, donde a igualdade é válida para $n=1$ (base de indução). Suponhamos que seja válida para um $n=k$ (hipótese de indução):

\begin{equation*}
	1^2 + 2^2 + ... + k^2 = \frac{k(k+1)(2k+1)}{6}
\end{equation*}

\noindent Acrescentando o sucessor (k+1) em ambos lados da igualdade, obtemos 

\begin{eqnarray*}
	1^2 + 2^2 + ... + k^2 + (k+1)^2 &=& \frac{(k+1)(k+2)(2k+3)}{6} \\
	 &=& \frac{k(k+1)(2k+1)}{6} + (k+1)^2 
\end{eqnarray*}

\noindent de modo que a igualdade também vale para $n = k + 1$. Pelo PIF, a igualdade vale para todo número natural $n \geq 1$.

\begin{exem}
	Demonstrar que para todo inteiro positivo n,
	\begin{equation*}
	1^3 + 2^3 + ... + k^3 = (1 + 2 + ... + n)^2
	\end{equation*}
\end{exem}
\noindent Solução: Neste caso vemos que $P(1): 1^3 = (1)^2$ é válido para $n = 1$ (base de indução). Observamos que o termo do segundo membro da igualdade mostrado no exemplo \ref{exemplo-ind1} pode ser substituído por
\begin{equation*}
	1^3 + 2^3 + ... + n^3 = \left[\frac{n(n+1)}{2}\right]^2.
\end{equation*}
\noindent Suponhamos que seja verdadeiro para $n = k$ (hipótese de indução).
\begin{equation*}
1^3 + 2^3 + ... + k^3 = \left[\frac{k(k+1)}{2}\right]^2.
\end{equation*}
\noindent Acrescentando o sucessor $n = k + 1$ em ambos os lados da igualdade, obtemos
\begin{eqnarray*}
	1^3 + 2^3 + ... + k^3 + (k+1)^3 &=& \left[\frac{k(k+1)}{2}\right]^2 + (k+1)^3\\
	&=& \frac{k^2(k+1)^2}{4} + (k+1)^3 \\
	&=& \frac{k^2(k+1)^2 + 4(k+1)^3}{4} \\
	&=& \frac{(k+1)^2[k^2 + 4(k+1)]}{4} \\
	&=& \frac{(k+1)^2(k+2)^2}{4} \\
	&=& \left(\frac{(k+1)(k+2)}{2} \right)^2.
\end{eqnarray*}
\noindent Vemos que é válido para $n = k + 1$. Portanto, pelo PIF, a igualdade vale para todo número natural $n \geq 1$.

A segunda forma do PIF (às vezes chamada de princípio de indução forte), possuem as seguintes propriedades 
\begin{enumerate}
	\item (Base de Indução): $P(n_0)$ é verdadeira; e
	\item (Passo Indutivo): Se $P(k)$ é verdadeira para todo natural k tal que $n_0 \leq k \leq n$, então $P(n+1)$ também é verdadeira.
\end{enumerate}









O \textit{princípio da boa ordenação} (PBO) dos números naturais, afirma que todo subconjunto $A$ não vazio de $\mathbb{N}$ tem um elemento mínimo. Ou seja,
\begin{itemize}
	\item Se $A \subseteq \mathbb{N}$ e $A \neq \emptyset$, então existe $n_0 \in A$ tal que $n_0 \leq n$, $\forall n \in A$
\end{itemize}

\begin{exem}
	
\end{exem}

 
 
 
 	\section{Divisibilidade e Congruência}
 	
    	\subsection{Divisibilidade}

Nesta secção descreveremos algumas propriedades da divisão, existência e unicidade do quociente e do resto na divisão de inteiros. \\

\noindent \textbf{Definição:} Dado dois inteiros \textit{a} e \textit{b}, dizemos que \textit{a} divide \textit{b} ou que \textit{a} é divisor de \textit{b} ou ainda que \textit{b} é um múltiplo de \textit{a} e denotado 

\begin{align*}
  a|b
\end{align*}

\noindent se existir um inteiro \textit{c} tal que $b = ac$.    \\
    
\noindent \textbf{Proposição:} Se \textit{a, b} e \textit{c} são inteiros, $a|b$ e $b|c$, então $a|c$. 

\noindent \textbf{Demonstração:} Temos que $a|b$ e $b|c$, neste caso existem inteiros $k_1$ e $k_2$ que $b = k_1a$ e $c = k_2b$. Substituindo a igualdade de b na segunda equação, teremos $c = k_1k_2a$ o que implica que $a|c$. Esta proposição é chamada de \textit{"Transitividade"}. \\

\noindent \textbf{Proposição:} Se \textit{a, b, c, m} e \textit{n} são inteiros, 

Além das proposições que apresentamos a divisibilidade tem as seguintes propriedades:

\begin{enumerate}[label=(\roman*)]
	\item $n|n$ \\ \textbf{Demonstração:} Se $n|n$, então existe um inteiro $k = 1$ para ser válido a igualdade $n = 1n$.
	\item $d|n \Rightarrow ad | an$ \\ \textbf{Demonstração:} Se $d|n$, então n é múltiplo de d, ou seja, existe um $k$ fixo que $n = kd$. Multiplicando um inteiro qualquer $a$ nos membros desta equação, temos $an = akd$, o que implica $ad | an$. Logo $d|n \Rightarrow ad|an$.
	\item $ad|an$ e $a \neq 0 \Rightarrow d|n$ \\ \textbf{Demonstração:} Se $ad|an e n \neq 0$, então $an$ é múltiplo de $ad$, ou seja, $an = kad$ e sendo $a \neq 0$ dividimos os dois membros da equação por $a$, assim temos que $n = kd$, o que implica $d|n$.
	\item $1|n$ \\ \textbf{Demonstração:} Se $1|n$, então $n = 1k$, sendo válida apenas com inteiro fixo $k = n$, o que nos mostras que 1 divide qualquer inteiro.
	\item $n|0$ \\ \textbf{Demonstração:} Seja $n|0$, ou seja, $0 = nk$ 
	\item $d|n$ e $n \neq 0 \Rightarrow |d| \leq |n|$ \\ \textbf{Demonstração:} Seja $d|n$ e $n \neq 0$, então n é múltiplo de $d$, ou seja, $n = kd$ neste caso temos que $|d| < |n|$ ou $|d| = |n|$ para o caso $k=1$, o que implica que $|d| \leq |n|$.
	\item $d|n$ e $n|d \Rightarrow |d| = |n|$ \\ \noindent\textbf{Demonstração:} Temos que $d|n$ e $n|d$, então $n = k_1d$ e $d = k_2n$. Substituindo a igualdade de n na 2ª equaçao $d=k_1k_2d$, dividindo os membros por $d$, $1=k_1k_2$. Como 1 é elemento neutro no operador de multiplicação, implica que $|d| = |n|$.
	\item $d|n$ e $d \neq 0 \Rightarrow (n/d)|n$ \\ \textbf{Demonstração:} Temos que $d|n$ sendo que $d \neq 0$, então existe um $k$ fixo que $n = kd$, como $d \neq 0$ podemos dividir os membros da igualdade por $d$, $\frac{n}{d} = k$, substituindo a igualdade de k na equação $n = kd \Rightarrow n = \frac{n}{d}d$, o que nos leva a entender que $(n/d)|n$.
\end{enumerate}

    \subsection{O Algoritmo da Divisão}

No célebre livro VII dos "Elementos" de Euclides escrito aproximadamente 300 a.c. é enunciado o teorema de Eudoxius, que será uma ferramenta essencial para demonstrar o Algoritmo da divisão.

\noindent \textbf{Teorema de Eudoxius:} Dado dois inteiros \textit{a} e \textit{b}, $b \neq 0$, então ou $a$ é múltiplo de $b$ ou $a$ se encontra entre dois múltiplos consecutivos de $b$. Ou seja, correspondendo a cada par de inteiros $a$ e $b \neq 0$ existe um inteiro $q$ tal que, para $b > 0$,

\begin{equation*}
	qb \leq a < (q+1)b
\end{equation*}

\noindent e para $b<0$,

\begin{equation*}
	qb \leq a < (q-1)b
\end{equation*}

\noindent \textbf{Demonstração:}      



Podemos finalmente enunciar e provar o Algoritmo da Divisão

\noindent \textbf{Teorema:} Dado dois inteiros $a$ e $b$, $b>0$, existe um único par de inteiros $q$ e $r$ tais que 

\begin{equation*}
	a = qb + r, \ \ \text{com} \ \ 0 \leq r < b \ \ (r = 0 \Leftrightarrow b|a)
\end{equation*}   

\noindent Os números $q$ e $r$ são chamados, respectivamente, \textbf{quociente} e \textbf{resto} da divisão de $a$ por $b$.

\noindent \textbf{Demonstração:}


		\subsection{mdc, mmc e Algoritmo Euclidiano}    
        
		
O \textit{máximo divisor comum} de dois inteiros $a$ e $b$ ($a$ ou $b$ diferentes de $0$), é denotado por $(a,b)$, é o maior inteiro que divide $a$ e $b$.

\noindent \textbf{Teorema:}

\noindent \textbf{Demonstração:}
		
		
		\subsection{Teorema Fundamental da Aritmética}
		

  		\subsection{Congruência}

As bases teóricas sobre congruência ou aritmética modular teve início com os trabalhos do matemático suíço Leonhard Euler. Porém, no livro \textit{Disquisitiones Arithmeticae} publicado em 1801, por Carl Friedrich Gauss que tinha apenas 24 anos de idade, tornou mais explícita com as anotações, simbologia e definições usados até hoje.

\begin{cdefinicao}
	Sejam a, b e n $\in \mathbb{Z}$, dizemos que a e b são congruentes módulo n se os restos de sua divisão euclidiana por n são iguais. Escreve-se $a \equiv b \pmod{n}$, quando está relação é falsa, denotamos por $a \not\equiv b \pmod{n}$ e denominamos de incongruentes.
\end{cdefinicao}

\begin{exem}
	$11 \equiv 3 \ (mod \ 2)$ pois $2|(11-3)$.
\end{exem}

\begin{prop}	
	Sejam a, b e n $\in \mathbb{Z}$, temos que $a \equiv b \pmod{n}$ se, e somente se, existir um k $\in \mathbb{Z}$ tal que $a = b + km$.
\end{prop}

\noindent \textbf{Demonstração:} Se $a \equiv b \pmod{n}$, então $n|(a - b)$ o que implica a existência de um $k \in \mathbb{Z}$ tal que $a - b = kn$, isto é, $a = b + kn$. A recíproca é trivial.

Conforme \citeonline{landau2002teoria} todos os conceitos como "congruente", "equivalente", "igual" ou "similar"  devem satisfazer três propriedades chamadas reflexiva, simétrica e transitiva. Além desses, apresentamos mais propriedades importantes sobre congruência segundo \citeonline{primospasseios}.

\begin{prop}
	Se a, b, c e n $\in \mathbb{Z}$, $n \leq 0$, as seguintes propriedades são válidas:
	\begin{enumerate}
		\item (Reflexiva): $a \equiv a \pmod{n}$;
		\item (Simétrica): Se $a \equiv b \pmod{n}$, então $b \equiv a \pmod{n}$;
		\item (Transitiva): Se $a \equiv b \pmod{n}$ e $b \equiv c \pmod{n}$, então $a \equiv c \pmod{n}$;
		\item (Compatibilidade com a soma e diferença): Podemos somar e subtrair membro a mebro \\
		\\
			$ \left\{
			\begin{array}{l}
			a \equiv b \pmod{n} \\
			c \equiv d \pmod{n} \\
			\end{array} \right. \Longrightarrow 
			  \left\{
			 \begin{array}{l}
			 a + c \equiv b + d \pmod{n} \\
			 a + c \equiv b - d \pmod{n} \\
			 \end{array}
			  \right. $ \\ 
			  \\ Em particular, se $a \equiv b \pmod{n}$, então $ka \equiv kb \pmod{n}$ para todo $k \in \mathbb{Z}$.
		\item (Compatibilidade com o produto): Podemos multiplicar membro a membro \\
		\\
			$ \left\{
			\begin{array}{l}
			a \equiv b \pmod{n} \\
			c \equiv d \pmod{n} \\
			\end{array} \right. 
			\Longrightarrow 
			\left\{
			\begin{array}{l}
			ac \equiv bd \pmod{n} \\
			\end{array}
			\right. 
			$ \\
			\\ Em particular, se $a \equiv b \pmod{n}$, então $a^k \equiv b^k \pmod{n}$ para todo $k \in \mathbb{N}$.
		\item (Cancelamento): Se $mdc(c,n) = 1$, então 
			\begin{equation*}
				ac \equiv bc \pmod{n} \Longleftrightarrow a \equiv b \pmod{n}
			\end{equation*}
	\end{enumerate}
\end{prop}

\noindent \textbf{Demonstração:}

  		
	  		\subsubsection{Congruência Linear}
  		
  		\subsection{Teoremas de Euler, Fermat e Wilson}
	
	\section{Pseudoprimos e Teste de Primalidade}

O prefixo \textit{pseudo} é usado para marcar algo que superficial, imitação, engano - ou seja, parece ser uma coisa, mas é outra coisa. Neste caso pseudoprimos não números que apresentam propriedades de números primos, mas não são primos.
		
		\subsection{Números de Carmichael}
	
		\subsection{Teste de Primalidade}

	
	
		\subsection{Distribuição dos Números Primos}
		
		
%	\chapter{CONCLUSÕES}
\label{chap:conclusoes}


	
	%Elementos pós-textuais	
	\bibliography{elementos-pos-textuais/referencias}
%	\imprimirglossario	
	\imprimirapendices
		% Adicione aqui os apendices do seu trabalho
		\apendice{FUNÇÃO DELTA DE DIRAC}
\label{ap:delta-dirac}

		\apendice{INTEGRAL DO PRODUTO}
\label{ap:product-integral}



%		\apendice{Termo de Fiel Depositário}
\label{ap:termo-de-fiel-depositario}

\noindent \textbf{Pesquisa:} ANÁLISE DA MORTALIDADE INFANTIL COM MALFORMAÇÕES CONGÊNITAS.

\noindent Pelo presente instrumento que atende às exigências legais, a Sra. Maria Consuelo Martins Saraiva, ``fiel depositário'' com o cargo de Secretária Municipal de Saúde de Iracema, após ter tomado conhecimento do protocolo de pesquisa intitulado: ANÁLISE DA MORTALIDADE INFANTIL COM MALFORMAÇÕES CONGÊNITAS. Analisando a repercussão desse estudo no contexto da saúde pública e epidemiologia, autoriza Karla Maria da Silva Lima, enfermeira, aluna do Curso de Mestrado Acadêmico em Enfermagem da Universidade Estadual do Ceará (UECE), sob orientação do Prof. Dr. José Maria de Castro, da UECE, ter acesso aos bancos de dados do Sistema de Informação sobre Nascidos Vivos e do Sistema de Informação sobre Mortalidade da Secretaria Municipal de Saúde de Iracema, objeto deste estudo, e que se encontram sob sua total responsabilidade. Fica claro que o Fiel Depositário pode a qualquer momento retirar sua AUTORIZAÇÃO e ciente de que todas as informações prestadas tornar-se-ão confidenciais e guardadas por força de sigilo profissional, assegurando que os dados obtidos da pesquisa serão somente utilizados para estudo.	
%	\imprimiranexos
		% Adicione aqui os anexos do seu trabalho
%		\anexo{EXEMPLO DE ANEXO}
\label{anexo:EXEMPLO}

		
%		\anexo{Dinâmica das classes sociais}
\label{an:dinamica-das-classes-sociais}

\lipsum[14]
\index{AAA}

%	\imprimirindice

\end{document}
