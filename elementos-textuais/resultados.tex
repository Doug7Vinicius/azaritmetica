\chapter{RESULTADOS E DISCUSSÕES}
\label{chap:resultados}

  \section{Lógica Matemática}

Conforme \cite{de2002iniciaccao} a lógica matemática adota como regra fundamentais do pensamento os dois seguintes princípios (ou axiomas):

\begin{axioma}
\textbf{PRINCÍPIO DA NÃO CONTRADIÇÃO:} \textit{uma proposição não pode ser verdadeira e falsa ao mesmo tempo.}
\end{axioma}

\begin{axioma}
\textbf{PRINCÍPIO DO TERCEIRO EXCLUÍDO:} \textit{toda proposição ou é verdadeira ou falsa, isto é, verifica-se sempre um destes casos e nunca um terceiro.}
\end{axioma}
  
Em virtude desses dois princípios temos que a Lógica Matemática é uma lógica bivalente. Por definição, proposição significa uma oração declarativa, que tem sentido afirmativo completo.  

    \subsection{Proposições Simples e Compostos}
    
Definição: chama-se de proposição simples ou proposições atômicas aquela que não contém nenhuma outra proposição como parte integrante de si mesma. Normalmente representado por letras latina minúsculas: \textit{p, q, r}.

\noindent\textbf{Exemplo:} O número 25 é quadrado perfeito.

Definição: chama-se proposição composta ou proposição molecular aquela formada pela combinação de duas ou mais proposições. Representado por letras latinas maiúsculas: \textit{P, Q, R}.

\noindent\textbf{Exemplo:} O número $\pi$ é irracional e maior do que 4.

Usualmente as proposições compostas que são formadas pela combinação das proposições simples \textit{p, q, r,...}, escreve-se: $P(p,q,r,...)$
    
    \subsection{Conectivos}

Definição: chama-se conectivos palavras ou expressões que se usam para formar, interligar novas proposições a partir de outras. São conectivos usuais em lógica as palavras: "e", "ou", "não", "se... então... ", "... se e somente se...".

		\subsubsection{Negação}
		
\textbf{Definição:} negação de uma preposição \textit{p} representada por "não \textit{p}" cujo o valor lógico é verdadeiro quando \textit{p} é falso e vice-verso.

\begin{center}		
\begin{tabular}{@{ }c | c@{ }@{ }c}
	p & $\sim$ & p\\
	\hline 
	V & F & \\
	F & V & \\
\end{tabular}
\end{center}

		\subsubsection{Conjunção}

\textbf{Definição:} a conjunção de duas proposições \textit{p} e \textit{q} a proposição representada por "\textit{p} e \textit{q}, cujo o valor lógico é a verdade (\textbf{V}) quando ambas proposições são verdadeiras e falsa nos demais casos.

\begin{center}
\begin{tabular}{@{ }c@{ }@{ }c | c@{}@{ }c@{ }@{ }c@{ }@{ }c@{ }@{}c@{ }}
	p & q &  & p & $\land$ & q & \\
	\hline 
	V & V &  &  & V &  & \\
	V & F &  &  & F &  & \\
	F & V &  &  & F &  & \\
	F & F &  &  & F &  & \\
\end{tabular} 
\end{center}
	
		\subsubsection{Disjunção}

\textbf{Definição:} a disjunção de duas proposições \textit{p} e \textit{q}, representadas por "\textit{p} ou \textit{q}", cujo o valor lógico é a verdade quando ao menos uma das proposições é verdadeira e falsa quando ambas as preposições são falsas.

\begin{center}
\begin{tabular}{@{ }c@{ }@{ }c | c@{}@{ }c@{ }@{ }c@{ }@{ }c@{ }@{}c@{ }}
	p & q &  & p & $\lor$ & q & \\
	\hline 
	V & V &  &  & V &  & \\
	V & F &  &  & V &  & \\
	F & V &  &  & V &  & \\
	F & F &  &  & F &  & \\
\end{tabular}
\end{center}

		\subsubsection{Disjunção Exclusiva}
		
\textbf{Definição:} disjunção exclusiva de duas proposições \textit{p} e \textit{q} representada por "ou \textit{p} ou \textit{q}" ou "ou \textit{p} ou \textit{q}, mas não ambas", cujo o valor lógico é a verdade somente quando \textit{p} é verdadeiro ou \textit{q} é verdadeiro

\begin{center}
\begin{tabular}{@{ }c@{ }@{ }c | c@{}@{ }c@{ }@{ }c@{ }@{ }c@{ }@{}c@{ }}
	p & q &  & p & $\veebar$ & q & \\
	\hline 
	V & V &  &  & F &  & \\
	V & F &  &  & V &  & \\
	F & V &  &  & V &  & \\
	F & F &  &  & F &  & \\
\end{tabular}
\end{center}

		\subsubsection{Condicional}

\noindent \textbf{Definição:} Proposição condicional representado por "se p, então q, cujo o valor lógico é falsidade no caso que p é verdadeira e q é falso e verdade nos demais casos. Os valores lógicos de duas proposições, definido pela seguinte tabela-verdade:

\begin{center}
	\begin{tabular}{@{ }c@{ }@{ }c | c@{}@{ }c@{ }@{ }c@{ }@{ }c@{ }@{}c@{ }}
		p & q &  & p & $\to$ & q & \\
		\hline 
		V & V &  &  & V &  & \\
		V & F &  &  & F &  & \\
		F & V &  &  & V &  & \\
		F & F &  &  & V &  & \\
	\end{tabular}
\end{center}

Também se lê de uma das seguintes maneiras:

\begin{enumerate}[label=(\roman*)]
	\centering
	\item p é condição suficiente para q;
	\item q é condição necessária para p.
\end{enumerate}

Neste conectivo lógico é diz que p é o \textbf{antecedente} e q o \textbf{consequente}. O símbolo "$\to$" é chamado de símbolo de implicação.

		
		
		\subsubsection{Bicondicional}
		
\noindent \textbf{Definição:} Proposição bicondicional representada por "p se e somente se q", cujo o valor lógico é a verdade quando p e q são ambas verdadeiras ou ambas falsas, e falsidade nos demais casos. O valor lógico da bicondicional de duas proposições definidas pela seguinte tabela-verdade:

		

\begin{center}
	\begin{tabular}{@{ }c@{ }@{ }c | c@{}@{ }c@{ }@{ }c@{ }@{ }c@{ }@{}c@{ }}
		p & q &  & p & $\leftrightarrow$ & q & \\
		\hline 
		V & V &  &  & V &  & \\
		V & F &  &  & F &  & \\
		F & V &  &  & F &  & \\
		F & F &  &  & V &  & \\
	\end{tabular}
\end{center}

Simbolicamente, "$\Leftrightarrow$" também se lê de uma das seguintes maneiras:
\begin{enumerate}
	\item p é condição necessária e suficiente para q;
	\item q é condição necessária e suficiente para p.
\end{enumerate}
  
	\section{Conjuntos}
	


  	\section{Tipos de Demonstrações}
  
Resumidamente existem dois tipos de demonstrações em matemática: as demonstrações direta e demonstrações indiretas. Grande parte dos teoremas tem a estrutura "se \textit{P}, então \textit{Q}, onde \textit{P} e \textit{Q} são afirmações falsas ou verdadeiras. Chamamos de \textit{P} de hipótese ou premissa e \textit{Q} de tese ou conclusão.

    \subsection{Demonstrações Diretas}

O procedimento para uma demonstração direta é através de implicações lógicas encadeadas que levam $P$ diretamente a $Q$. Ou seja, de uma sentença $p \to q$ funciona da seguinte forma: assuma que o antecedente $p$ é verdade (hipótese) e deduza o consequente (tese) $q$.

\noindent \textbf{Exemplo 1:} Quaisquer dois quadrados perfeitos consecutivos diferem por um número ímpar.

\noindent \textbf{Demonstração:} Suponhamos que $a$ e $b$ sejam inteiros quadrados perfeitos consecutivos, ou seja, $a = n^2$ e $b = (n+1)^2$. Queremos mostrar que eles diferem por um número ímpar, ou seja, $b-a$ ou $a-b$ é um número ímpar. Como $a = n^2$ e $b = (n+1)^2$ então $b-a = (n+1)^2 - n^2 = n^2+2n+1-n^2 = 2n+1$. Portanto, $b-a$ é um número ímpar.

\noindent \textbf{Exemplo 2:} Sejam $A$ e $B$ conjuntos. Mostre que $A \subset B \Leftrightarrow \complement_B \subset \complement_A$.

\noindent \textbf{Demonstração:} Suponhamos que $A \subset B$. Então um elemento $x \in \complement_B$ não pode pertencer ao conjunto $B$, principalmente não pertencer a $A$. Logo $x \in \complement_B \Rightarrow x \in \complement_A$, ou seja, $\complement_B \subset \complement_A$. Semelhantemente pela propriedade reflexiva, $\complement(\complement_A) = A$, temos $\complement_B \subset \complement_A$ então, $\complement(\complement_A) \subset \complement(\complement_B)$, obtemos $A \subset B$.

    \subsection{Demonstrações Indiretas}

Existem dois tipos de demonstrações indireta que podemos usar para estabelecer uma afirmação condicional da forma $P \Rightarrow Q$: demonstração por absurdo e demonstração por contra-positiva.

\noindent \textbf{Demonstração por absurdo:} Na demonstração por absurdo nós assumimos que a hipótese $P$ é verdadeira, mas nesse caso supomos que a conclusão $Q$ é falsa. O objetivo é mostrar que a combinação da validade da hipótese $P$ com a não validade da tese $Q$ produz um resultado absurdo. Dessa forma segue que $Q$ é verdadeira.

\noindent \textbf{Exemplo:} Mostre que $\sqrt{2} \notin \mathbb{Q}$. Isso é o mesmo que dizer: Se $x \in \mathbb{R}, \ x > 0 $ e $ x^2 = 2$, então $x \notin \mathbb{Q}$.

\noindent \textbf{Demonstração:} Sabemos que um número $r \in \mathbb{R}$ é dito racional se existem inteiros $p, \ q$ sendo $q \neq 0$ tais que $r = \frac{p}{q}$.
  
  
    \section{Indução Finita e Boa Ordenação}

Uma propriedade básica dos números naturais e uma ferramenta indispensável na demonstração de muitos teoremas é: o Princípio da Indução Finita (PIF) que é divida em duas formas.

Seja $P(n)$ uma propriedade do número natural $n$, por exemplo:
\begin{itemize}
  \item $1+2+...+n=\frac{n(n+1)}{2}$;
  \item $1^2 + 2^2 + ... + n^2 = \frac{n(n+1)(2n+1)}{6}$;
  \item $1^3 + 2^3 + ... + n^3 = (1 + 2 + ... + n)^2$;
  \item $F_1 + F_2 + ... + F_n = F_{n+2} - 1$;
\end{itemize}

Para provar que a sentença aberta $P(n)$ é válida para todo natural $n \geq n_0$ é necessário utilizar o \textit{Princípio da Indução Finita} (PIF), que é um dos axiomas só o conjunto dos números naturais possuem. O PIF consiste em verificar duas etapas:
\begin{enumerate}
	\item (Base de Indução): $P(n_0)$ é verdadeira;
	\item (Passo Indutivo): Se $P(n)$ é verdadeira para algum número natural $n \geq n_0$, então $P(n+1)$ também é verdadeira.
\end{enumerate}

\begin{exem}\label{exemplo-ind1}
	Demonstrar que para todo inteiro positivo n,
\begin{equation*}
	1 + 2 + ... + n = \frac{n(n+1)}{2}
\end{equation*}
\end{exem}

\noindent Solução: Sabemos que para $P(1): 1 = \frac{1(1+1)}{2}$, onde a igualdade é verdadeira para $n = 1$ (base de indução). 
Suponhamos que seja verdadeiro para um $n = k$ (hipótese de indução):
\begin{equation*}
	1 + 2 + ... + k = \frac{k(k+1)}{2}.
\end{equation*}
\noindent Somando $k+1$ em ambos os lados da igualdade, obtemos
\begin{eqnarray*}
	1 + 2 + ... + k + (k+1) &=& \frac{k(k+1)}{2} + (k+1) \\
	&=& \frac{(k+1)(k+2)}{2},
\end{eqnarray*}
\noindent neste caso a igualdade também é válida para $n = k+1$. Pelo PIF, a igualdade vale para todo número natural $n \geq 1$.

\begin{exem}
	Demonstrar que para todo inteiro positivo n,
	\begin{equation*}
	1^2 + 2^2 + ... + k^2 = \frac{n(n+1)(2n+1)}{6}
	\end{equation*}
\end{exem}

\noindent Solução: Observamos que para $P(1): 1^2 = \frac{1(1+1)(2 \cdot 1 + 1)}{6}$, donde a igualdade é válida para $n=1$ (base de indução). Suponhamos que seja válida para um $n=k$ (hipótese de indução):

\begin{equation*}
	1^2 + 2^2 + ... + k^2 = \frac{k(k+1)(2k+1)}{6}
\end{equation*}

\noindent Acrescentando o sucessor (k+1) em ambos lados da igualdade, obtemos 

\begin{eqnarray*}
	1^2 + 2^2 + ... + k^2 + (k+1)^2 &=& \frac{(k+1)(k+2)(2k+3)}{6} \\
	 &=& \frac{k(k+1)(2k+1)}{6} + (k+1)^2 
\end{eqnarray*}

\noindent de modo que a igualdade também vale para $n = k + 1$. Pelo PIF, a igualdade vale para todo número natural $n \geq 1$.

\begin{exem}
	Demonstrar que para todo inteiro positivo n,
	\begin{equation*}
	1^3 + 2^3 + ... + k^3 = (1 + 2 + ... + n)^2
	\end{equation*}
\end{exem}
\noindent Solução: Neste caso vemos que $P(1): 1^3 = (1)^2$ é válido para $n = 1$ (base de indução). Observamos que o termo do segundo membro da igualdade mostrado no exemplo \ref{exemplo-ind1} pode ser substituído por
\begin{equation*}
	1^3 + 2^3 + ... + n^3 = \left[\frac{n(n+1)}{2}\right]^2.
\end{equation*}
\noindent Suponhamos que seja verdadeiro para $n = k$ (hipótese de indução).
\begin{equation*}
1^3 + 2^3 + ... + k^3 = \left[\frac{k(k+1)}{2}\right]^2.
\end{equation*}
\noindent Acrescentando o sucessor $n = k + 1$ em ambos os lados da igualdade, obtemos
\begin{eqnarray*}
	1^3 + 2^3 + ... + k^3 + (k+1)^3 &=& \left[\frac{k(k+1)}{2}\right]^2 + (k+1)^3\\
	&=& \frac{k^2(k+1)^2}{4} + (k+1)^3 \\
	&=& \frac{k^2(k+1)^2 + 4(k+1)^3}{4} \\
	&=& \frac{(k+1)^2[k^2 + 4(k+1)]}{4} \\
	&=& \frac{(k+1)^2(k+2)^2}{4} \\
	&=& \left(\frac{(k+1)(k+2)}{2} \right)^2.
\end{eqnarray*}
\noindent Vemos que é válido para $n = k + 1$. Portanto, pelo PIF, a igualdade vale para todo número natural $n \geq 1$.

A segunda forma do PIF (às vezes chamada de princípio de indução forte), possuem as seguintes propriedades 
\begin{enumerate}
	\item (Base de Indução): $P(n_0)$ é verdadeira; e
	\item (Passo Indutivo): Se $P(k)$ é verdadeira para todo natural k tal que $n_0 \leq k \leq n$, então $P(n+1)$ também é verdadeira.
\end{enumerate}









O \textit{princípio da boa ordenação} (PBO) dos números naturais, afirma que todo subconjunto $A$ não vazio de $\mathbb{N}$ tem um elemento mínimo. Ou seja,
\begin{itemize}
	\item Se $A \subseteq \mathbb{N}$ e $A \neq \emptyset$, então existe $n_0 \in A$ tal que $n_0 \leq n$, $\forall n \in A$
\end{itemize}

\begin{exem}
	
\end{exem}

 
 
 
 	\section{Divisibilidade e Congruência}
 	
    	\subsection{Divisibilidade}

Nesta secção descreveremos algumas propriedades da divisão, existência e unicidade do quociente e do resto na divisão de inteiros. \\

\noindent \textbf{Definição:} Dado dois inteiros \textit{a} e \textit{b}, dizemos que \textit{a} divide \textit{b} ou que \textit{a} é divisor de \textit{b} ou ainda que \textit{b} é um múltiplo de \textit{a} e denotado 

\begin{align*}
  a|b
\end{align*}

\noindent se existir um inteiro \textit{c} tal que $b = ac$.    \\
    
\noindent \textbf{Proposição:} Se \textit{a, b} e \textit{c} são inteiros, $a|b$ e $b|c$, então $a|c$. 

\noindent \textbf{Demonstração:} Temos que $a|b$ e $b|c$, neste caso existem inteiros $k_1$ e $k_2$ que $b = k_1a$ e $c = k_2b$. Substituindo a igualdade de b na segunda equação, teremos $c = k_1k_2a$ o que implica que $a|c$. Esta proposição é chamada de \textit{"Transitividade"}. \\

\noindent \textbf{Proposição:} Se \textit{a, b, c, m} e \textit{n} são inteiros, 

Além das proposições que apresentamos a divisibilidade tem as seguintes propriedades:

\begin{enumerate}[label=(\roman*)]
	\item $n|n$ \\ \textbf{Demonstração:} Se $n|n$, então existe um inteiro $k = 1$ para ser válido a igualdade $n = 1n$.
	\item $d|n \Rightarrow ad | an$ \\ \textbf{Demonstração:} Se $d|n$, então n é múltiplo de d, ou seja, existe um $k$ fixo que $n = kd$. Multiplicando um inteiro qualquer $a$ nos membros desta equação, temos $an = akd$, o que implica $ad | an$. Logo $d|n \Rightarrow ad|an$.
	\item $ad|an$ e $a \neq 0 \Rightarrow d|n$ \\ \textbf{Demonstração:} Se $ad|an e n \neq 0$, então $an$ é múltiplo de $ad$, ou seja, $an = kad$ e sendo $a \neq 0$ dividimos os dois membros da equação por $a$, assim temos que $n = kd$, o que implica $d|n$.
	\item $1|n$ \\ \textbf{Demonstração:} Se $1|n$, então $n = 1k$, sendo válida apenas com inteiro fixo $k = n$, o que nos mostras que 1 divide qualquer inteiro.
	\item $n|0$ \\ \textbf{Demonstração:} Seja $n|0$, ou seja, $0 = nk$ 
	\item $d|n$ e $n \neq 0 \Rightarrow |d| \leq |n|$ \\ \textbf{Demonstração:} Seja $d|n$ e $n \neq 0$, então n é múltiplo de $d$, ou seja, $n = kd$ neste caso temos que $|d| < |n|$ ou $|d| = |n|$ para o caso $k=1$, o que implica que $|d| \leq |n|$.
	\item $d|n$ e $n|d \Rightarrow |d| = |n|$ \\ \noindent\textbf{Demonstração:} Temos que $d|n$ e $n|d$, então $n = k_1d$ e $d = k_2n$. Substituindo a igualdade de n na 2ª equaçao $d=k_1k_2d$, dividindo os membros por $d$, $1=k_1k_2$. Como 1 é elemento neutro no operador de multiplicação, implica que $|d| = |n|$.
	\item $d|n$ e $d \neq 0 \Rightarrow (n/d)|n$ \\ \textbf{Demonstração:} Temos que $d|n$ sendo que $d \neq 0$, então existe um $k$ fixo que $n = kd$, como $d \neq 0$ podemos dividir os membros da igualdade por $d$, $\frac{n}{d} = k$, substituindo a igualdade de k na equação $n = kd \Rightarrow n = \frac{n}{d}d$, o que nos leva a entender que $(n/d)|n$.
\end{enumerate}

    \subsection{O Algoritmo da Divisão}

No célebre livro VII dos "Elementos" de Euclides escrito aproximadamente 300 a.c. é enunciado o teorema de Eudoxius, que será uma ferramenta essencial para demonstrar o Algoritmo da divisão.

\noindent \textbf{Teorema de Eudoxius:} Dado dois inteiros \textit{a} e \textit{b}, $b \neq 0$, então ou $a$ é múltiplo de $b$ ou $a$ se encontra entre dois múltiplos consecutivos de $b$. Ou seja, correspondendo a cada par de inteiros $a$ e $b \neq 0$ existe um inteiro $q$ tal que, para $b > 0$,

\begin{equation*}
	qb \leq a < (q+1)b
\end{equation*}

\noindent e para $b<0$,

\begin{equation*}
	qb \leq a < (q-1)b
\end{equation*}

\noindent \textbf{Demonstração:}      



Podemos finalmente enunciar e provar o Algoritmo da Divisão

\noindent \textbf{Teorema:} Dado dois inteiros $a$ e $b$, $b>0$, existe um único par de inteiros $q$ e $r$ tais que 

\begin{equation*}
	a = qb + r, \ \ \text{com} \ \ 0 \leq r < b \ \ (r = 0 \Leftrightarrow b|a)
\end{equation*}   

\noindent Os números $q$ e $r$ são chamados, respectivamente, \textbf{quociente} e \textbf{resto} da divisão de $a$ por $b$.

\noindent \textbf{Demonstração:}


		\subsection{mdc, mmc e Algoritmo Euclidiano}    
        
		
O \textit{máximo divisor comum} de dois inteiros $a$ e $b$ ($a$ ou $b$ diferentes de $0$), é denotado por $(a,b)$, é o maior inteiro que divide $a$ e $b$.

\noindent \textbf{Teorema:}

\noindent \textbf{Demonstração:}
		
		
		\subsection{Teorema Fundamental da Aritmética}
		

  		\subsection{Congruência}

As bases teóricas sobre congruência ou aritmética modular teve início com os trabalhos do matemático suíço Leonhard Euler. Porém, no livro \textit{Disquisitiones Arithmeticae} publicado em 1801, por Carl Friedrich Gauss que tinha apenas 24 anos de idade, tornou mais explícita com as anotações, simbologia e definições usados até hoje.

\begin{cdefinicao}
	Sejam a, b e n $\in \mathbb{Z}$, dizemos que a e b são congruentes módulo n se os restos de sua divisão euclidiana por n são iguais. Escreve-se $a \equiv b \pmod{n}$, quando está relação é falsa, denotamos por $a \not\equiv b \pmod{n}$ e denominamos de incongruentes.
\end{cdefinicao}

\begin{exem}
	$11 \equiv 3 \ (mod \ 2)$ pois $2|(11-3)$.
\end{exem}

\begin{prop}	
	Sejam a, b e n $\in \mathbb{Z}$, temos que $a \equiv b \pmod{n}$ se, e somente se, existir um k $\in \mathbb{Z}$ tal que $a = b + km$.
\end{prop}

\noindent \textbf{Demonstração:} Se $a \equiv b \pmod{n}$, então $n|(a - b)$ o que implica a existência de um $k \in \mathbb{Z}$ tal que $a - b = kn$, isto é, $a = b + kn$. A recíproca é trivial.

Conforme \citeonline{landau2002teoria} todos os conceitos como "congruente", "equivalente", "igual" ou "similar"  devem satisfazer três propriedades chamadas reflexiva, simétrica e transitiva. Além desses, apresentamos mais propriedades importantes sobre congruência segundo \citeonline{primospasseios}.

\begin{prop}
	Se a, b, c e n $\in \mathbb{Z}$, $n \leq 0$, as seguintes propriedades são válidas:
	\begin{enumerate}
		\item (Reflexiva): $a \equiv a \pmod{n}$;
		\item (Simétrica): Se $a \equiv b \pmod{n}$, então $b \equiv a \pmod{n}$;
		\item (Transitiva): Se $a \equiv b \pmod{n}$ e $b \equiv c \pmod{n}$, então $a \equiv c \pmod{n}$;
		\item (Compatibilidade com a soma e diferença): Podemos somar e subtrair membro a mebro \\
		\\
			$ \left\{
			\begin{array}{l}
			a \equiv b \pmod{n} \\
			c \equiv d \pmod{n} \\
			\end{array} \right. \Longrightarrow 
			  \left\{
			 \begin{array}{l}
			 a + c \equiv b + d \pmod{n} \\
			 a + c \equiv b - d \pmod{n} \\
			 \end{array}
			  \right. $ \\ 
			  \\ Em particular, se $a \equiv b \pmod{n}$, então $ka \equiv kb \pmod{n}$ para todo $k \in \mathbb{Z}$.
		\item (Compatibilidade com o produto): Podemos multiplicar membro a membro \\
		\\
			$ \left\{
			\begin{array}{l}
			a \equiv b \pmod{n} \\
			c \equiv d \pmod{n} \\
			\end{array} \right. 
			\Longrightarrow 
			\left\{
			\begin{array}{l}
			ac \equiv bd \pmod{n} \\
			\end{array}
			\right. 
			$ \\
			\\ Em particular, se $a \equiv b \pmod{n}$, então $a^k \equiv b^k \pmod{n}$ para todo $k \in \mathbb{N}$.
		\item (Cancelamento): Se $mdc(c,n) = 1$, então 
			\begin{equation*}
				ac \equiv bc \pmod{n} \Longleftrightarrow a \equiv b \pmod{n}
			\end{equation*}
	\end{enumerate}
\end{prop}

\noindent \textbf{Demonstração:}

  		
	  		\subsubsection{Congruência Linear}
  		
  		\subsection{Teoremas de Euler, Fermat e Wilson}
	
	\section{Pseudoprimos e Teste de Primalidade}

O prefixo \textit{pseudo} é usado para marcar algo que superficial, imitação, engano - ou seja, parece ser uma coisa, mas é outra coisa. Neste caso pseudoprimos não números que apresentam propriedades de números primos, mas não são primos.
		
		\subsection{Números de Carmichael}
	
		\subsection{Teste de Primalidade}

	
	
		\subsection{Distribuição dos Números Primos}
		
		